\documentclass{article}
\usepackage[utf8]{inputenc}
\usepackage[T1]{fontenc}
\usepackage{lmodern}
\usepackage[english]{babel}
\usepackage{amsmath,amssymb,amsfonts,amsthm,mathtools,sansmath,wasysym}
\usepackage{cleveref}
\usepackage{tikz-cd}
\usepackage{url}

%Commands
\newcommand{\contradiction}{\lightning}
\newcommand{\N}{\mathbb{N}}
\newcommand{\J}{\mathfrak{J}}
\newcommand{\Z}{\mathbb{Z}}
\newcommand{\m}{\mathfrak{m}}
\newcommand{\p}{\mathfrak{p}}
\newcommand{\nr}{\mathfrak{N}}
\newcommand{\im}{\mathrm{im}}
\newcommand{\lead}{\mathrm{lead}}
\newcommand{\codeg}{\mathrm{deg}}
\makeatletter
\def\fall#1{\forall #1\@ifnextchar\bgroup{\,,\fall}{:\,}}
\makeatother
\renewcommand{\S}[1]{\mathbb{S}^{#1}}
\newcommand{\T}[1]{\mathbb{T}^{#1}}
\newcommand{\inv}[1]{{#1}^{-1}}
\renewcommand{\O}{\mathcal{O}}
\newcommand{\R}{\mathbb{R}}
\newcommand{\pr}{\mathrm{pr}}
\newcommand{\id}{\mathrm{id}}
\newcommand{\foukof}[1]{\hat{\varphi}_{{#1}}}
\newcommand{\e}[1]{\mathsf{e}^{#1}}
\renewcommand{\i}{\mathsf{i}}
\newcommand{\claim}
{\underline{\textit{Claim:}}\hspace{0,2cm}}
\newcommand{\subclaim}[1]
{

	\vspace*{0,2cm}
	\textit{Claim}({#1}):
}
\newcommand{\subqed}[1]{\hfill\textsf{qed}({#1})}
\newcommand{\subproof}{

\noindent\textit{proof}.\hspace{0,1cm}
}
%Aufgaben-Command
\newcommand{\aufgabe}[1]{
{
	\vspace*{0.5cm}
	\noindent\textsf{\textbf{Exercise #1}}
	\vspace*{0.2cm}

}
}
%unteraufgabe
\newcommand{\unteraufgabe}[1]{
{
	\textsf{(#1)}
}
}
%Teilaufgabe
\newcommand{\teilaufgabe}[1]{
{       

	\noindent\hspace*{0,1 cm}\textbf{#1)}
}
}

\newcommand{\induktionsanfang}{
{
	\vspace*{0.1cm}
	\noindent
	\textsf{Induktionsanfang:}
}
}

\newcommand{\induktionsschritt}{
{
	\vspace*{0.1cm}
	\noindent
	\textsf{Induktionsschritt:}
}
}
%Umgebungen
\newtheorem{thm}{Theorem}[section] 
\theoremstyle{definition}
\newtheorem{defn}[thm]{Definition}
\newtheorem*{silentdefn}{Definition}
\newtheorem*{silentthm}{Theorem}
\newtheorem*{silentlem}{Lemma}
\theoremstyle{plain}
\newtheorem{cor}[thm]{Korollar}
\newtheorem{lem}[thm]{Lemma}
\newtheorem{propo}[thm]{Proposition}
\newtheorem{axiom}[thm]{Axiom}
\theoremstyle{remark}
\newtheorem{remark}[thm]{Remark}
\newtheorem*{silentremark}{Remark}
\newtheorem{example}[thm]{Example}

\title{Homework No.2}
\author{Philipp Stassen}
\begin{document}
\maketitle
\aufgabe1 Let $T_A$ be the endomorphism of $\T2$ by the matrix $A = 
\begin{pmatrix}
	2 &0 \\
	0 &3
\end{pmatrix}$

We have that $\det(A) = 6$ and the eigenvalues are $x_1 = 2$ resp. $x_2= 3$. Clearly on $A^{-1} = 
\begin{pmatrix}
	\frac{1}{2} &0\\
	0&\frac{1}{3}
\end{pmatrix}$.

\subclaim1 $h(f)\geq \log2+\log3$
\subproof
Let $\varepsilon < \frac{1}{6}$. As the largest eigenvalue of $A$ is $x_2 = 3$ we have for any point $x\in \T2$ that $T_A\upharpoonright B(x,\varepsilon)$ is injective.
Then we can cover $\T2$ by $d_n$-open Balls $B_n(p_i,\varepsilon)$. We have that
\begin{align}
	B_n(p_i,\varepsilon) = \bigcap_{k=0}^{n-1}T_{A}^{-k}B(T_{A}^k(p_i),\varepsilon)
\end{align}
and therefore we can proceed similar as in \emph{Example 3.15}\footnote{Dynamical systems an introduction Chapter 3.4.3 - L. Barreira, C. Valls}. Note that here $T_A^{-k}$ denotes the inverse function on the restricted domain, not the preimage. We have $T_A^{-1}=T_{A^{-1}}$ (on a restricted domain such that $T_A$ is invertible).

There exists an constant $C>0$ such that the area of $B_n(p_i,\varepsilon)$ is smaller then $C 2^{-n}3^{-n}\varepsilon^2$. 

%To be a bit more precise, actually already $T_{A}^{-(n-1)}B(T_{A}^{n-1}(p_i),\varepsilon)$ has an area that is smaller than $C2^-{n}3^{-n}\varepsilon^2$.
Thus, $M(n,\varepsilon)\geq \frac{1}{C 2^{-n}3^{-n}\varepsilon^2}$ and by taking limits we gets we get
\begin{align}
	h(f)&=\lim_{\varepsilon\to 0}\liminf_{n\to\infty}\frac{1}{n}\log M(n,\varepsilon)\\
	&\geq \lim_{\varepsilon\to 0}\liminf_{n\to\infty}\frac{1}{n}\log \left(\frac{1}{C 2^{-n}3^{-n}\varepsilon^2}\right)\\
										  &=\lim_{\varepsilon\to 0}\liminf_{n\to\infty}\left(\frac{1}{n}\log \left(\frac{1}{C \varepsilon^2}\right)+\frac{1}{n}n\log2+\frac{1}{n}n\log3\right) \\
	&=\log2+\log3
\end{align}
\subqed1
\subclaim2 $h(f)\leq \log2+\log3$
\subproof
On the other hand we consider a partition of $\T2$ by parallelograms $P_i$ with side length $3^{-n}\varepsilon$ and $2^{-n}\varepsilon$ along the eigendirections, i.e. rectangles as the eigendirections are $(1,0)$ and $(0,1)$. 

There is a constant $D>1$ (I think $\sqrt2$ would be a precise choice) such that the $d_n$-Diameter of each $P_i$ is smaller than $D\varepsilon$.
Furthermore we have that the area of each $P_i$ is at least $D^{-1}2^{-n}3^{-n}\varepsilon^2$.
Therefore we can give the bound.
\begin{align}
	C(n,D\varepsilon)\leq \frac{1}{D^{-1}2^{-n}3^{-n}\varepsilon^2} = D\varepsilon^{-2}2^n3^n
\end{align}
Therefore, we can conclude
\begin{align}
	h(f)&=\lim_{\epsilon\to 0}\lim_{n\to\infty}\frac{1}{n}\log C(n,\varepsilon) \\
	    &=\lim_{D\epsilon\to 0}\lim_{n\to\infty}\frac{1}{n}\log C(n,D\varepsilon)\\
	    &\leq\lim_{\epsilon\to 0}\lim_{n\to\infty}\frac{1}{n}\log (D\varepsilon^{-2}2^n3^n) \\
	    &=\lim_{\epsilon\to 0}\lim_{n\to\infty}(\frac{1}{n}\log (D\varepsilon^{-2})+\frac{1}{n}\log(2^n)+\frac{1}{n}\log(3^n) \\
	    &= \log2 + \log 3
\end{align}
\subqed2

Hence, we can conclude that $h(f)=\log2+\log3$. \qed

\aufgabe2
Let $f:X\to X$ be a continuous map of a compact metric space, let $k\in\N$. 
\teilaufgabe{a}
Let $d_{n,f}=d_n$ and $N_f(n,\varepsilon)=N(n,\varepsilon)$.

\claim $d_{n,f^k}(x,y)\leq d_{nk,f}(x,y)$, and thus $N_{f^k}(n,\varepsilon)\leq N_f(nk,\varepsilon)$.
\begin{proof}
	We have hat 
	\begin{align}
		d_{n,f^k}(x,y) &= \max\{d\left(f^{ki}(x),f^{ki}(y)\right)|i\leq n-1\} \\
			       &\leq\max\{d\left(f^{i}(x),f^{i}(y)\right)|i\leq nk-1\} \\ 
			       &= d_{nk,f}(x,y)
	\end{align}
	Now it follows that if $p_1,...,p_n$ are points such that $d_{n,f^k}(p_i,p_j)\geq\varepsilon$ for $i\neq j$, we also have $d_{nk,f}(p_i,p_j)\geq d_{n,f^k}(p_i,p_j) \geq\varepsilon$. Hence, as 
	\begin{align}
		N_f(nk,\varepsilon)&= \max\{m\in\N| \exists p_1 ... p_m : \bigwedge_{i\neq j} d_{nk}(p_i,p_j)\geq\varepsilon\} \\
				 &\geq\max\{m\in\N| \exists p_1 ... p_m : \bigwedge_{i\neq j} d_{n,f^k}(p_i,p_j)\geq\varepsilon\}\\
				 &= N_{f^k}(n,\varepsilon)
	\end{align}.
\end{proof}
\teilaufgabe{b} \claim $h(f^k)\leq kh(f)$.
\begin{proof}
	We have that
\begin{align}
	\frac{\log N_{f^k}(n,\varepsilon)}{n} \leq \frac{\log N_f(kn,\varepsilon)}{n} = \frac{\log N_f(m(n),\varepsilon)}{\frac{1}{k}m(n)} = \frac{k\log N_f(m(n),\varepsilon)}{m(n)}
\end{align}
where $m(n):=kn$ is a sequence in $\N$. Note that 
\begin{align}
	\limsup_{n\to\infty}\frac{k\log N_f(m(n),\varepsilon)}{m(n)}\leq \limsup_{n\to\infty}\frac{k\log N_f(n,\varepsilon)}{n}
\end{align} 
as the first is a subsequence of the latter one. This lets us conclude
	\begin{align}
		h(f^k) &= \lim_{\varepsilon\to0}\limsup_{n\to\infty}\frac{\log N_{f^k}(n,\varepsilon)}{n} \\
		       &\leq\lim_{\varepsilon\to0}\limsup_{n\to\infty}\frac{\log N_f(kn,\varepsilon)}{n} \\
		       &\leq\lim_{\varepsilon\to0}\limsup_{n\to\infty}\frac{k\log N_f(m(n),\varepsilon)}{m(n)} \\
		       &\leq\lim_{\varepsilon\to0}\limsup_{n\to\infty}\frac{\log N_f(n,\varepsilon)}{n}k = kh(f)
	\end{align}
\end{proof}
\teilaufgabe{c} \claim $\lim_{\varepsilon\to 0}\limsup_{n\to\infty}\frac{1}{n}\log N_f(nk,\varepsilon)\leq h(f^k)$
\begin{proof}
	\subclaim1 For every $\varepsilon > 0$ exists $\delta(\varepsilon) \in (0,\varepsilon)$ such that $d(x,y)<\delta(\varepsilon)\Longrightarrow d_{k,f}(x,y)<\varepsilon$ for every $x,y\in X$. 
	\subproof As $X$ is compact and $f$ is continuous it follows that $f$ and all $f^n$ are uniformly continuous. Hence we can define $\delta(\varepsilon):=\min_{0\leq i\leq k}\{\delta(\varepsilon, i)\,|\, \fall{x,y\in X}d(x,y)<\delta(\varepsilon,i) \Longrightarrow d_{i,f}(x,y)<\varepsilon\}$. In case that $\delta(\varepsilon)>\varepsilon$ we redefine it to be $\frac{\varepsilon}{2}$. Therefore, we can assume$\delta(\varepsilon)\in(0,\varepsilon)$.
	\subqed1

	\subclaim2 $d_{nk}(x,y) \geq \varepsilon \Longrightarrow d_{n,f^k}(x,y)\geq \delta(\varepsilon)$
	\subproof
	Note that $d_{nk}(x,y)=\max\{d_k(f^{ik}(x),f^{ik}(y)) \,|\, 0\leq i\leq n-1 \}$.
	Hence, we know that if $d_{nk}(x,y)\geq \varepsilon$ then $d_k(f^{ik}(x),f^{ik}(y))\geq \varepsilon$ for some $i\leq n-1$ and therefore $d(f^{ik}(x),f^{ik}(y))>\delta(\varepsilon)$ by the uniform continuity of $f$. Thus, $d_{n,f^k}(x,y)\geq d(f^{ik}(x),f^{ik}(y))\geq \delta(\varepsilon)$ as claimed. \subqed2

	\subclaim3 $N(nk,\varepsilon)\leq N_{f^k}(n,\delta(\varepsilon))$
	\subproof Let $p_1,...,p_n$ be a $nk,\varepsilon$ - spanning set for $f$, i.e. there $d_{nk}(p_i,p_j)\geq \varepsilon$ for all $i\neq j\leq n$. By Claim(2) we also have $d_{n,f^k}(p_i,p_j)\geq \varepsilon$ for all $i\neq j\leq n$. Hence $p_1,...,p_n$ is an $n,\delta(\varepsilon)$ - spanning set for $f^k$. \subqed3 \medskip

	Now we can conclude the \underline{claim}
\end{proof}
\end{document}
